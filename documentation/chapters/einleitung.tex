\chapter{Einleitung}
Die vorliegende Arbeit stellt einen Debugger für einen CISC-Prozessor vor. Hierbei handelt es sich um den in \cite{Stroetmann2007} vorgestellten Prozessor \emph{Mic-1}:
\begin{quote}
Die Abkürzung CISC steht hier für \emph{complex instruction set computer} und drückt aus, dass die Maschinensprache wesentlich umfangreicher ist als bei einem RISC-Prozessor.
\end{quote}

Im Gegensatz zu einem RISC-Prozessor sind die Maschinenbefehle in einem CISC-Prozessor nicht in Hardware implementiert, sondern werden durch einen Mikroprogramm-Speicher definiert. Dieser wird in Mikro-Assembler programmiert und ermöglicht es, die Maschinenbefehle flexibel zu steuern.

Der Debugger soll dazu die Entwicklung von Mikro-Assembler- und Assembler-Programmen für den CISC-Prozessor erleichtern.