\chapter{Werkzeuge}
Wie sind die bisher vorgestellten Funktionen des Debuggers implementiert? Mit diesem Teil der Arbeit möchte ich diese Frage beantworten. Mein Ziel ist, die Implementierung und Struktur des Debuggers so nachvollziehbar zu beschreiben, dass Sie in der Lage sind den Debugger weiterzuentwickeln.

Der Debugger soll auch nach Ende dieser Arbeit weiterentwickelt werden können. Daher stelle ich in diesem Kapitel vor, welche Werkzeuge ich zur Entwicklung des Debuggers nutze und wie Sie diese zur Mitarbeit nutzen können. Dies ist nicht als Bedienungsanleitung zu verstehen. Ich bin der Ansicht, dass es das Verständnis der Implementierung und meiner Arbeit steigert, wenn geklärt ist, welche Werkzeuge ich wie nutze.

\section{Versionskontrollsystem}
Das Projekt ist bei github % TODO
gehostet und ist somit mit \git versioniert.

\subsection{Code auschecken}
\subsection{Feature implementieren}
\subsection{Bug fixen}
\section{Build-Management}
\subsection{Code kompilieren}
\subsection{Tests ausführen}
\subsection{Abhängigkeiten ändern}
\section{Entwicklungsumgebung}
\subsection{Integration von git und maven}
\subsection{Navigation im Code}
\subsection{Finden von Referenzen}